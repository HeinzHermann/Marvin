\documentclass{article}

%\usepackage[asinew]{inputenc}

\usepackage[backend=biber]{biblatex}
%\usepackage[autostyle, german=quotes]{csquotes}
\bibliography{bibliography}


\usepackage[utf8]{inputenc} %Zeichencodierung
\usepackage[T1]{fontenc}
\usepackage{amsmath, amsfonts,amssymb,amsthm}
\usepackage{graphicx} %bastel Graphiken in Latex documente

\usepackage{hyperref} %unten erklaerungen (vielleicht?)
%\label{aussagekraefige Namen} %irgendwas mit veweisen
%


\title{Lesson 2 - Zitations und Gliederungen}
\author{Marvin Glaser}
%\date{<insert>}

\begin{document}

\tableofcontents
\newpage
\maketitle

\section{Einleitung}
was passiert in dieser Arbeit

\section{Method}
wie bin ich zu meinem Ergebnis gekommen

    \subsection{Gundlagen der Graphentheorie}
    Graphen sind toll.
    
        \subsubsection{Kuerzeste Wege in Graphen}
        Sind das hier und immer gut fuer die Laufzeit
        
        \subsubsection{Der Dijkstra-Algotithmus}
        Den verwende ich, weil der die voll schnell findet und ywar so.
    
    \subsection {Grundlagen von Python}
    Python ist eine Programmiersprache.

\section{Ergebnis}
was ist mein Ergebnis?

\section{Diskussion und Asblilck}
wie sind diese Ergebnisse im wissenschaftlichen Gesamtkontext einzuordnen?

\end{document}

