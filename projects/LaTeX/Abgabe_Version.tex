\documentclass{scrartcl}

\usepackage[utf8]{inputenc} %enables utf8 encoding
%\usepackage[latin1]{inputenc}
\usepackage[ngerman]{babel}
\usepackage[T1]{fontenc}
\usepackage{ragged2e}       %used for text alignment
\usepackage{amsmath, amsfonts,amssymb,amsthm} %stuff???
\usepackage{endnotes}
%Umlaute and shit  Ä/ä, Ö/ö, Ü/ü  ß
\usepackage{hyperref}
\usepackage{blindtext}
\usepackage{lipsum}
\usepackage{mwe}
\usepackage{graphicx}
\usepackage{cite}
\usepackage[backend=biber]{biblatex}
\bibliography{text_bib}


\renewcaptionname{ngerman}{\figurename}{Abb.}



\title{Ein Dokument zum Üeben}
\author{Marvin Glaser, 4424114}
\date{42.02.0815}

\begin{document}
 \maketitle
\section{Einleitung}
 Dieses Dokument dient dazu, das Durchgenommene des \LaTeX{}-Einführungskurses - oder besser des \glqq LaTex-Einführungskurses \footnote{alternativ könnte man diesen Einschub bezüglich der Schreibweise von \LaTeX{} auch als Fußnote machen} \grqq - zu üben und zu vertiefen. \par
 Der obrige Satz ist übrigens ein Beispiel für eine {\itshape Alliteration}. Was genau ist - also eine \hyperref[Definition]{Definition} - werden wir am Mittwoch dem Domkument hinzufügen; das Dokument wird also im Verlauf des Kurses noch ein bisschen wachsen...
 \newline
 \newline
 
% \begin{flushleft}
 \label{Definition}\noindent \textbf{Definition 1.1.} Eine \textit{Alliteration} ein literarisches Stilmittel oder ein rethorisches Ele\-ment, bei der benachbarte Wörter den gleichen Anfangslaut (Anlaut) besitzen.
% \end{flushleft}

    
 \begin{itemize}
     
 \item[] Diese Definition ist dem zugehörigen Wikipedia-Artikel \cite{testing}
 entnommen.\newline Mit \LaTeX{} kann man u.a. folgende Sachen machen:
 
 \item Aufzählungen und Auflistungen
        \begin{enumerate}
            \item Aufzählungen
            \item Auflistungen
            \item auch mit mehreren Ebenen / verschachtelt
                \begin{itemize}
                 \item was nicht unbedingt immer sinvoll ist, bis zu einer beliebigen Tiefe zu machen. Auch sehr lange Unterpunkte können komisch wirken. Oder eine Extraebene, in der dann nur ein Unterpunkt auftaucht...
                \end{itemize}
            \item eine Tabelle
            \item ein paar Formeln und mathematische Ausdrücke
            \item Textformatierungen
        \end{enumerate}
 \end{itemize}\par
 
Eher schlechter Stil ist {\tiny im Übrigen}, {\Huge ständig} die {\Large Schriftgröße} zu ändern oder Text will{\Huge kürli}ch {\itshape kursiv}, \textbf {fett}, \underline{unterstrichen} oder Ko\underline{m\textbf{bi}{\itshape na}t\textbf{io}n}en daraus zu setzen.

\newpage

\begin{table}
\begin{tabular}{c | c | l}
 Ein mathematischer Ausdruck & Leicht verändert & Bemerkungen \\
 \hline
 \tiny{$\frac{1}{1 + \frac{1}{1 + \frac{1}{1 + \frac{1}{1 + ...}}}}$} % groesse anpassen
 & \large{$\frac{1}{1 + \frac{1}{1 + \frac{1}{1 + \frac{1}{1 + ...}}}} $} % groesse anpassen
 & ein Kettenbruch, si\-ehe \newline auch \glqq goldener Schnitt \grqq , meist bezeichnet mit $\phi$  \\
 
 
 \hline
 $\int_{a}^{b} f(x) dx = [F(x)]_{a}^{b} = F(b) - F(a)$
 & 
 & siehe auch Hauptsatz der Differenzial- und In\-tegralrechnung \\
 \hline
 $1 + 2 + ... + n = \sum_{i = 1}^{n} i = \frac{n\cdot(n + 1)}{2}$
 &
 & Einfachsummenformel\\
 \hline
\end{tabular}

 \caption{Eine Tabelle mit mathematischen Ausdr\"ucken}
 \label{table1}
\end{table}

\section{Der Mathemodus, Tabellen \& mehr}

%Umlaute and shit  Ä/ä, Ö/ö, Ü/ü  ß

Neben den Formeln in \hyperref[table1]{Tabelle 1} ein paar weitere Sachen im Mathemodus:

\begin{itemize}
 
 \item Abschnittsweise definierte Funktionen \newline
 \begin{center}
 $
 f(x) = \begin{cases}
          (\frac{1}{2}) & \text{falls}\ x \in \mathbb{R} \setminus \mathbb{Q} \\
          (\frac{1}{3} & \text{falls}\ x \in \mathbb{Q}
         \end{cases}
 $
 \end{center}
 \item Daferner kann man beisielsweise Wurzelzeichen $\sqrt[3]{2 + (3*2)}$ und Binominalkoeffizienten $\binom{n}{k}$ machen
 
\end{itemize}



\textbf{Bemerkung 1.} Um das Dokument etwas größer zu machen, f\"ugen wir an dieser Stelle mithilfe
der Pakete \textit{blindtext} \footnote{Die ausf\"uhrliche Dokumentation des blindtext Paketes findet man \"ubrigens
\href{http://tug.ctan.org/tex-archive/macros/latex/contrib/blindtext/blindtext.pdf}{hier}.} \textit{libsum} Blindtext ein.
\footnote{Sollte bei Ihnen nicht der gleiche Text entstehen, ist das nicht schlimm - das Dokument muss am Schluss nicht v\"ollig exakt wie das Original sein.}
Zun\"achst ersteres mit dem Befehl \textbackslash{}blindmathtrue und
\textbackslash{}blindtext:\newline

\blindmathtrue
\blindtext
\newline

\textbf{Bemerkung 2.} Und dann nutzen wir das lipsum-Paket und den Befehl \textbackslash lipsum[2]
\newpage

\lipsum[2]
\newline

\noindent \textbf{Bemerkung 3.} Im Quellcode folgt nun ein eingef\"ugtes Bild
\footnote{Ihrer Wahl - Sie k\"onnen auch das Paket mwe
laden und als in includegraphics den \glqq Dateinamen\grqq \textit{example-image} benutzen - das Ergebnis sehen Sie hier im
tweiten [sic] Bild von \hyperref[abb2]{Abb. 2}}
(dessen Breite gerade $\frac{3}{4}$, also dem 0,75-fachen der Textbreite \textbackslash textwidth entspricht) in einer figure-Umgebung. Wo das dann letztlich erscheint, sind wir alle gespannt.

 
\begin{figure}[h]
 \begin{center}
  \includegraphics[width=0.75\textwidth]{example-image}
 \end{center}
  \caption[Ein Bild aus Florenz]{Hier steht eine lange Bildunterschrift, dass es sich hierbei um ein Foto aus Florenz - gemacht von Ponte Vecchio aus - handelt; im Abbildungsverzeichnis hingegen steht ein k\"urzerer Text}
  \label{abb1}
\end{figure}

Wer nichts findet, wie man die Abbildungsunterschriften statt \textit{Abbildung} die Ab\-k\"urzung \textit{Abb} .
hinbekommt oder wie man auch die Referenzen (mit \textbackslash autoref) die andere Schreibweise erzeugen lassen kann,
kann sich mal \href{https://texwelt.de/wissen/fragen/991/wie-kann-ich-die-von-autoref-verwendete-bezeichnung-fur-abbildungen-andern}{diesen Beitrag} anschauen. Wem das momentan zu kompliziert zum Nachvollziehen ist, der darf gerne auch \textit{Abbildungen} stehen lassen.
\newline

\noindent \textbf{Theorem 1.} \textit{Selbst \"uber Florenz sind manchmal Wolken zu sehen}.
\newline \textit{Beweis}. Das sieht man leicht in \hyperref[abb1]{Abb. 1} 
\newline
\newline

Zum Abschluss des Dokuments f\"ugen wir nur noch zwei Bilder nebeneinadner ein. Das ginge zum Einen mittels subfigures in
der figure-Umgebung. Dabei werden die captions aber nummeriert. Stattdessen benutzen wir in \hyperref[abb2]{Abb. 2} eine
\href{https://www.namsu.de/Extra/befehle/Minipage.html}{minipage}:
\newline

\begin{figure}[h]
\begin{minipage}[c]{0.5\textwidth}
    \includegraphics[width=\textwidth]{example-image}
\end{minipage}
\begin{minipage}[c]{0.5\textwidth}
    \includegraphics[width=\textwidth]{example-image}
\end{minipage}

\caption{Zwei Bilder nebeneinadner}
\label{abb2}
\end{figure}

\listoftables

\listoffigures

%\bibliographystyle{plain}

\end{document}
