\documentclass{scrartcl}

\usepackage[utf8]{inputenc} %enables utf8 encoding
%\usepackage[latin1]{inputenc}
\usepackage[ngerman]{babel}
\usepackage[T1]{fontenc}
\usepackage{ragged2e}       %used for text alignment
\usepackage{amsmath, amsfonts,amssymb,amsthm} %stuff???
\usepackage{endnotes}
%Umlaute and shit  Ä/ä, Ö/ö, Ü/ü  ß
\usepackage{hyperref}



\title{Ein Dokument zum Üeben}
\author{Marvin Glaser, 4424114}
\date{42.02.0815}

\begin{document}
 \maketitle
\section{Einleitung}
 Dieses Dokument dient dazu, das Durchgenommene des \LaTeX{}-Einführungskurses - oder besser des \glqq LaTex-Einführungskurses \footnote{alternativ könnte man diesen Einschub bezüglich der Schreibweise von \LaTeX{} auch als Fußnote machen} \grqq - zu üben und zu vertiefen. \par
 Der obrige Satz ist übrigens ein Beispiel für eine {\itshape Alliteration}. Was genau ist - also eine \hyperref[Definition]{Definition} - werden wir am Mittwoch dem Domkument hinzufügen; das Dokument wird also im Verlauf des Kurses noch ein bisschen wachsen...
 \newline
 \newline
 
% \begin{flushleft}
 \label{Definition}\noindent \textbf{Definition 1.1.} Eine \textit{Alliteration} ein literarisches Stilmittel oder ein rethorisches Ele\-ment, bei der benachbarte Wörter den gleichen Anfangslaut (Anlaut) besitzen.
% \end{flushleft}

    
 \begin{itemize}
     
 \item[] Diese Definition ist dem zugehörigen Wikipedia-Artikel 1 %hier REFERENZ einfuegen!!
 entnommen.\newline Mit \LaTeX{} kann man u.a. folgende Sachen machen:
 
 \item Aufzählungen und Auflistungen
        \begin{enumerate}
            \item Aufzählungen
            \item Auflistungen
            \item auch mit mehreren Ebenen / verschachtelt
                \begin{itemize}
                 \item was nicht unbedingt immer sinvoll ist, bis zu einer beliebigen Tiefe zu machen. Auch sehr lange Unterpunkte können komisch wirken. Oder eine Extraebene, in der dann nur ein Unterpunkt auftaucht...
                \end{itemize}
            \item eine Tabelle
            \item ein paar Formeln und mathematische Ausdrücke
            \item Textformatierungen
        \end{enumerate}
 \end{itemize}\par
 
Eher schlechter Stil ist {\tiny im Übrigen}, {\Huge ständig} die {\Large Schriftgröße} zu ändern oder Text will{\Huge kürli}ch {\itshape kursiv}, \textbf {fett}, \underline{unterstrichen} oder Ko\underline{m\textbf{bi}{\itshape na}t\textbf{io}n}en daraus zu setzen.

\newpage

\begin{tabular}{c | c | l}
 Ein mathematischer Ausdruck & Leicht verändert & Bemerkungen \\
 \hline
 \tiny{$\frac{1}{1 + \frac{1}{1 + \frac{1}{1 + \frac{1}{1 + ...}}}}$} % groesse anpassen
 & \large{$\frac{1}{1 + \frac{1}{1 + \frac{1}{1 + \frac{1}{1 + ...}}}} $} % groesse anpassen
 & ein Kettenbruch, si\-ehe \newline auch \glqq goldener Schnitt \grqq , meist bezeichnet mit $\phi$  \\
 
 
 \hline
 $\int_{a}^{b} f(x) dx = [F(x)]_{a}^{b} = F(b) - F(a)$
 & 
 & siehe auch Hauptsatz der Differenzial- und In\-tegralrechnung \\
 \hline
 $1 + 2 + ... + n = \sum_{i = 1}^{n} i = \frac{n\cdot(n + 1)}{2}$
 &
 & Einfachsummenformel\\
 \hline
\end{tabular}
\newline

\section{Der Mathemodus, Tabellen \& mehr}

%Umlaute and shit  Ä/ä, Ö/ö, Ü/ü  ß

Neben den Formeln in \hyperref[table1]{Tabelle 1} ein paar weitere Sachen im Mathemodus:

\begin{itemize}
 
 \item Abschnittsweise definierte Funktionen \newline
 \begin{center}
 $
 f(x) = \begin{cases}
          (\frac{1}{2}) & \text{falls}\ x \in \mathbb{R} \textbackslash \mathbb{Q} \\
          (\frac{1}{3} & \text{falls}\ x \in \mathbb{Q}
         \end{cases}
 $
 \end{center}
 \item Daferner kann man beisielsweise Wurzelzeichen $\sqrt[3]{2 + (3*2)}$ und Binominalkoeffizienten $\binom{n}{k}$ machen
 
\end{itemize}



\textbf{Bemerkung 1.} Um das Dokument etwas größer zu machen, f\"ugen wir an dieser Stelle mithilfe
der Pakete \textit{blindtext}$^{\hyperref[blindtext]{2}}$ und \textit{libsum} Blindtext ein.
$^{\hyperref[blind]{3}}$ Zun\"achst ersteres mit dem Befehl \textbackslash{}blindmathtrue und
\textbackslash{}blindtext:\newline

\blindtext %???
\par Dies ist ein Blindtext zum testen von Textausgaben. Wer diesen Text liest ist selbst Schuld.
$\sin^{2}{(\alpha)} + \cos^{2}{(\beta)} = 1$. Der Text gibt lediglich den Grauwert der Schrift an
\textit{E = mc}$^{2}$. %HIER VERBESSERN!!
Ist das wirklich so? Ist es gleichg\"ultig ob ich schreibe: \glqq Dies ist ein Blindtext\grqq oder \glqq Huardest gefburn\grqq ? Kjift - mitnichten! Ein Blindtext bietet mir wichtige Informationen.
\Blindtext %????


\end{document}
