\documentclass{scrartcl}

\usepackage[utf8]{inputenc} %Zeichencodierung
\usepackage[T1]{fontenc}
\usepackage{amsmath, amsfonts,amssymb,amsthm}

\title{testtitel}
\author{ich}


\begin{document}

\maketitle

Code fuer Anmeldung EMKS2018

Das ist das erste Document -- ist es nicht schoen?

Zur Demonstration von Slebentrennungen hier nun ein langes Wort: Donaudampfschiffahrtsgesellschaftskpitaensmuetze\newline
Donau\-dampf\-schiff\-fahrts\-gesell\-schafts\-kapitaens\-muetze\newline
%\underline{Demonstration}\newline
%\textit{Demonstration}\newline
%\emph{Demonstration}\newline
%\itshape Demonstration\newline
%{\itshape Demonstration}\newline
%\itshape Demonstration \normalfont \newline
%\textbf{Demonstration}\newline
%{\bfseries Demonstration}\newline
%\textit{\textbf{Demonstration}}\newline
%\newline\newline
%\Large{Demonstration}\newline
%{\Large Demonstration}\newline
%\Large Demonstration \normalsize\newline
%\Large Demonstration \LARGE \newline
%{\Huge Demonstration}\newline
%{\tiny Demonstration}\newline

\textbackslash \ $\backslash \{$ \& \%


\begin{itemize}
\item Erstes Item
\item Weiteres Item
\end{itemize}

\begin{tabular}{lc|r}
 1 & 2 & 3\\
 bla & blub & qweryz\\
 \hline
\end{tabular}
\newline \newline

a a\enspace a\quad a\qquad

%$\sum_{i=3}^{3}{i} = 6$
%\[\matchcal{P}(M)\coloneqq\{N \mid \text{N ist Teilmenge von } M\}\] %error
\[cos \phi \neq \text{cos} \phi \neq \cos \phi\]
\newline
\[0.5 = \frac{1}{2}\]

\[3*5\neq 3\times 5 \neq 3\cdot5\]

\end{document}

