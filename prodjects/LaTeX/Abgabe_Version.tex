\documentclass{scrartcl}

\usepackage[utf8]{inputenc} %enables utf8 encoding
\usepackage[T1]{fontenc}
\usepackage{ragged2e}       %used for text alignment
\usepackage{amsmath, amsfonts,amssymb,amsthm} %stuff???

%Umlaute and shit  Ä/ä, Ö/ö, Ü/ü  ß


\title{Ein Dokument zum Üeben}
\author{Marvin Glaser, 4424114}
\date{42.02.0815}

\begin{document}
 \maketitle

 Dieses Dokument dient dazu, das Durchgenommene des \LaTeX-Einführungskurses - oder besser des ``LaTex-Einführungskurses'' - zu üben und zu vertiefen. \par
 Der obrige Satz ist übrigens ein Beispiel für eine {\itshape Alliteration}. Was genau ist - also\newline eine Definition - werden wir am Mittwoch dem Domkument hinzufügen; das Dokument wird also im Verlauf des Kurses noch ein bisschen wachsen... \par
 Heute begnügen wir uns damit, die folgenden Sachen auszuprobieren:
 
 \begin{itemize}
  \item Aufzählungen und Auflistungen
        \begin{enumerate}
            \item Aufzählungen
            \item Auflistungen
            \item auch mit mehreren Ebenen / verschachtelt
                \begin{itemize}
                 \item was nicht unbedingt immer sinvoll ist, bis zu einer beliebigen Tiefe zu machen. Auch sehr lange Unterpunkte können komisch wirken. Oder eine Extraebene, in der dann nur ein Unterpunkt auftaucht...
                \end{itemize}
            \item eine Tabelle
            \item ein paar Formeln und mathematische Ausdrücke
            \item Textformatierungen
        \end{enumerate}
 \end{itemize}\par
 
Eher schlechter Stil ist {\tiny im Übrigen}, {\Huge ständig} die {\Large Schriftgröße} zu ändern oder Text will{\Huge kürli}ch {\itshape kursiv}, \textbf {fett}, \underline{unterstrichen} oder Ko\underline{m\textbf{bi}{\itshape na}t\textbf{io}n}en daraus zu setzen.
 
 
\end{document}
