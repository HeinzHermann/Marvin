\documentclass[german]{uebung}

\usepackage{uebung-meta}
\usepackage{enumitem}
\usepackage{amssymb}
\usepackage{multicol}
%%%%%%%%%%%%%%%%%%%%%%%%%%%%%%%%%%%%%%%%%%%%%%%%%%%%%%%%%%%%%%%%%%%%%%%%%%%%
% README
%%%%%%%%%%%%%%%%%%%%%%%%%%%%%%%%%%%%%%%%%%%%%%%%%%%%%%%%%%%%%%%%%%%%%%%%%%%%

% How to use this:
% 1. Add your data to uebung-meta.sty (you only need to do this once)
% 2. Copy this file and name it something useful
% 3. Set the assignment to the right value
% 4. Use the exercise enviroment to separate your solutions for the different exercises from each other.

% DO NOT CHANGE uebung.cls! If you need more packages just add a new \usepackage somewhere in this file before the \begin{document}

% The commands
% \div and \grad are provided by uebung.cls

% The environment "exercise" takes one parameter (the exercise number). 
% This way you can skip exercises if you like. Example:
% 
% \assignment{3}
% \begin{exercise}{8}
% ...
% \end{exercise}
% 
% The solution to exercise 3.8 (3rd assignment, 8th exercise) goes where 
% the dots are.

% If the total page number shows up as ?? in the footer you need to compile a second time.

% Which assignment is this?
\assignment{2}


\begin{document}

\begin{exercise}{1}
	\begin{enumerate}[label=(\alph*)]
		\item
	%Der Massenzufluss an Salz in das Gef\"a{\ss} pro Zeiteinheit sei definiert als $m_{in} = rc_0$.
	%Weiterhin sei der Massenabflu{\ss} an Salz definiert als $m_{out} = rc$, mit $c = c(t)$.
	Die Menge an Salz im Gef\"a{\ss} definiert als% $m = Vc$, mit $m = m(t)$, woraus sich eine
			zeitliche {\"A}nderung der Salzmasse im Gef\"a{\ss} von $m\dot = V $%c\dot$ ergibt.
	Diese l\"asst sich au{\ss}erdem darstellen als $m\dot = rc_0 - rc$.\\
	Kombiniert man diese Gleichungen, so erh\"alt man:

		%\begin{align}
		%			& m\dot	&= m\dot	\\
		%	\leftrightarrow & Vc\dot&= rc_0 - rc	\\
		%	\leftrightarrow	& c\dot &= \frac{rc_0}{V} - \frac{rc}{V}\\
		%\end{align}

		\item b
	\end{enumerate}
\end{exercise}

\begin{exercise}{2}
	part2
\end{exercise}

\begin{exercise}{3}
	part3
\end{exercise}


\begin{exercise}{4}
	part4
\end{exercise}


\begin{exercise}{5}
	part5
\end{exercise}

\end{document}
