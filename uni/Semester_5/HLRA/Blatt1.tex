\documentclass{article}

\usepackage[english]{babel}
\usepackage{hyperref}
\usepackage{graphicx}
\usepackage{listings}
\usepackage[printwatermark]{xwatermark}
\usepackage{xcolor}
\usepackage{todonotes}
\lstset{language=C}

% Top and Bottom Margin:  1  1/2"; Right and Left Margin:  1  1/2"
\setlength{\topmargin}{0in}
\setlength{\oddsidemargin}{0.5in}
\setlength{\textwidth}{5.5in}
\pagestyle{empty}
\setlength{\parskip}{0.2in}

\title{H 00: The Template}
\author{
Marvin Glaser \\ \texttt{4424114} \\ \href{mailto:marvin.glaser91@gmail.com}{marvin.glaser91@gmail.com}
% \and   
%Student2 \\ \texttt{7654321} \\ \href{mailto:stud2@email.com}{stud2@email.com}
%\and   
%Student3 \\ \texttt{0000000} \\ \href{mailto:stud3@email.com}{stud3@email.com}
}

%\newwatermark[allpages=true,color=gray!20,angle=55,scale=3,xpos=0,ypos=0]{Homework Template}

\begin{document}
\maketitle



\section{Task 1}

a)

b)

c)

d)




\section{Task 2}

a)
The peak performance of a system is categorized by the total usage of all availabe ressources
the system has to offer. At this point the systems only limitation are of physical limitations
(example: time information needs to "travel" between different compartments of the system)

b)
Operations per clock cycle (OPC)
Operations per second (OPS)
Parallel processors (PP)
FLoating point operations (FPO)
processor clock time (PCT)
Number of x (\#x)

calculation:

$
OPC = \#PP * \#ALU(per PP) * \#FPO(per ALU)
    = 512 *     8	 *     3 
    = 12288
$


{-->} 12288 Operations per clock cycle


$
OPS = #OPC * PCT
    = 12288 * 1*10^9 * 1/s
    = 12.288*10^12
$

12.288*10^12 operations per second

c)
It is almost impossible to archieve peak performace on any system. Reasons include the lack of
perfect parallelization, dependencies between different operations (operation B needs to wait
on operation A to finish, etc), waiting times between calculations (a processor needs to wait
for information that is needed to continue a process; expl. cash fault) and many more.


\section{Task 3}

a)
14.2 ms are linear
--> 57.6 can be parellized

$57.6 / 32 = 1.8 ms$ (since 32 processors are used)

--> $14.2 ms + 1.8 ms = 16 ms$ are needed to solve the problem

$72/16 = 4$

--> Speedup is 4


b)
$
72 * x + 0.25 * 72 * (1-x) = 32

72 * (x + 0.25 * (1-x)) = 32
72 * (x + 0.25 - 0.25x) = 32
72 * (0.75x + 0.25) = 32
48x + 16 = 32
48x = 16
x = 1/3
$

1/3 der Gesamtzeit (24ms) wird zum initialisieren benoetigt


c)
%REWORK!!!
$
maximum speedup with infinite amouts of processors
(infinite perfectly parallel working processors reduce the parallel runntime to effectively zero)
= normal runtime / (initialization time + paralellized time)
= 72 / (24 + lim_{x->0} x) = 72 / (24 + 0) = 3
$


if an infinite number of processors run perfecly parallel then only the initialization time
remains. Therefore the minimal runtime of the program is 24 ms
dumb boss sells before he thinks :(

d)
72 * 1/6 + (1/32) * 72 * 5/6 = x

72 * (1/6 + 5/192) = x
72 * (32/192 + 5/192) = x
72 * (37/192) = x
2664/192 = x
13.875 = x

--> With the new algorithm the program fnishes in less then 14 ms. The company is saved! YAY!!!





------------------------------------



\end{document}
