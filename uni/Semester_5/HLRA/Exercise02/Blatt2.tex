\documentclass{article}

\usepackage[english]{babel}
\usepackage{hyperref}
\usepackage{graphicx}
\usepackage{listings}
\usepackage[printwatermark]{xwatermark}
\usepackage{xcolor}
\usepackage{todonotes}
\usepackage{enumitem}
%\usepackage{enumerate}
\lstset{language=C}

% Top and Bottom Margin:  1  1/2"; Right and Left Margin:  1  1/2"
\setlength{\topmargin}{0in}
\setlength{\oddsidemargin}{0.5in}
\setlength{\textwidth}{5.5in}
\pagestyle{empty}
\setlength{\parskip}{0.2in}

\title{Hochleistungsrechnerarchitektur Blatt X}
\author{
Marvin Glaser \\ \texttt{4424114} \\ \href{mailto:marvin.glaser91@gmail.com}{marvin.glaser91@gmail.com}
% \and   
%Student2 \\ \texttt{7654321} \\ \href{mailto:stud2@email.com}{stud2@email.com}
%\and   
%Student3 \\ \texttt{0000000} \\ \href{mailto:stud3@email.com}{stud3@email.com}
}

%\newwatermark[allpages=true,color=gray!20,angle=55,scale=3,xpos=0,ypos=0]{Homework Template}

\begin{document}
\maketitle



\section{Task 1}

\begin{enumerate}[label=(\alph*)]
	
\item Cache associativity refers to different ways of storing blocks of  information in a given cache.
	Cashes can be of the types 'directed map', meaning that a given block of information 
	can be placed in one specific line of the cache. 'n-way associative', meaning a given
	block of information can be placed in a set of \textit{n} different cache blocks. Finally a
	cache can be 'fully associative', meaning that a cache block can be placed in any of the
	caches cache lines.
	
\item The cache in the given figure is 2-way associative. This is easy to understand by the fact
	that there are two given cache blocks with 16 cache lines each.

\item Sets of a cache refer to the number of cache lines a given adress can have. The nuber of sets
	directly correlates with the associativity of the cache. 

\item In the given cache their is a number of 16 sets with a size of 2.

\item A cache miss referce to the case when a processor requests information from one of its
	associated cache, but the cache does not contain the requested information. In this case
	the information needs to be fetched from the next bigger source of data.

\item 
	
\end{enumerate}



\section{Task 2}





\section{Task 3}







\end{document}
