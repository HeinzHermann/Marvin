\documentclass{article}

\usepackage[english]{babel}
\usepackage{hyperref}
\usepackage{graphicx}
\usepackage{listings}
\usepackage[printwatermark]{xwatermark}
\usepackage{xcolor}
\usepackage{todonotes}
\usepackage{datetime}
\usepackage{enumitem}
\usepackage{amsmath}
\lstset{language=C}

% Top and Bottom Margin:  1  1/2"; Right and Left Margin:  1  1/2"
\setlength{\topmargin}{0in}
\setlength{\oddsidemargin}{0.5in}
\setlength{\textwidth}{5.5in}
\pagestyle{empty}
\setlength{\parskip}{0.2in}

\title{High Performance Computing (2019/2020), Exercise 01}
\author{
Marvin Glaser \\ \texttt{4424114} \\ \href{mailto:marvin.glaser91@gmail.com}{marvin.glaser91@gmail.com}
% \and   
%Student2 \\ \texttt{7654321} \\ \href{mailto:stud2@email.com}{stud2@email.com}
%\and   
%Student3 \\ \texttt{0000000} \\ \href{mailto:stud3@email.com}{stud3@email.com}
}

\begin{document}
\maketitle


% taks 1
\section{Task}

\begin{enumerate}[label=(\alph*)]

\item The CDC 6600 was able to perform three megaFLOPS
($3*10^6$ floating point operations) per second and therefore was the worlds fastest computer
from 1964 to 1969.

Source: \\
\href{https://en.wikipedia.org/wiki/CDC_6600}{Wikipedia: CDC 6600}


\item The currently fastest supercomputer known to public is the Summit (or OLCF-4). Its fastest
measured performance using the 'LINPACK Benchmarks suite' was 148.6 petaFLOPs ($10^15$ FLOPs).
Its theoretical maximum peak performance is 200.795 petaFLOPs.

Source:
Numbers taken from:\\ 
\href{https://en.wikipedia.org/wiki/Summit_(supercomputer)}{Wikipeia, Summit}
\href{https://en.wikipedia.org/wiki/TOP500}{Wikipedia, TOP500}



\item The 'LINPACK Benchmarks' tool was used to determine the peak performance of the Summit
supercomputer. The tool measures a systems floating point computing power by measureing how fast
the system measures a 'dense n by n system of linear equations'.

Source: \\
\href{https://en.wikipedia.org/wiki/TOP500}{Wikipedia, TOP500}
\href{https://en.wikipedia.org/wiki/LINPACK_benchmarks}{LINPACK Benchmarks}


\item The fastest german supercomputer (the SuperMUC) has reacently fallen on place number 8 on the
TOP500 list.

Source:
\href{https://en.wikipedia.org/wiki/TOP500}{Wikipedia, TOP500}


\item Other possible rankings depend on the criteria that are looked at. For measuring pure performance
FLOPs are a very good system. Possible other rankings could involve power usage per FLOPs. Since
supercomputers tend to drain a lot of power a ranking like this could promote more efficient
architecture desing.
\end{enumerate}

% taks 2
\section{Task}

\begin{enumerate}[label=(\alph*)]


\item The peak performance of a system is categorized by the total usage of all availabe ressources
the system has to offer. At this point the systems only limitation are of physical limitations\\
(example: time information needs to "travel" between different compartments of the system)


\item Operations per clock cycle (OPC)\\
Operations per second (OPS)\\
Parallel processors (PP)\\
FLoating point operations (FPO)\\
processor clock time (PCT)\\
Number of x (\#x)

\begin{align*}
	OPC &= \#PP * \#ALU(per PP) * \#FPO(per ALU)\\
	    &= 512 *     8	 *     3 \\
	    &= 12288
\end{align*}


$\rightarrow$ 12288 Operations per clock cycle

\begin{align*}
	OPS &= \#OPC * PCT \\
	&= 12288 * 10^{9} * \frac{1}{s} \\
	&= 12.288*10^{12}
\end{align*}

$\rightarrow 12.288*10^{12}$ operations per second


\item It is almost impossible to archieve peak performace on any system. Reasons include the lack of
perfect parallelization, dependencies between different operations (operation B needs to wait
on operation A to finish, etc), waiting times between calculations (a processor needs to wait
for information that is needed to continue a process; expl. cash fault) and many more.
\end{enumerate}

% taks 3
\section{Task}

\begin{enumerate}[label=(\alph*)]

\item 14.2 ms are linear\\
$\rightarrow$ 57.6ms can be parellized

$\frac{57.6}{32} = 1.8 ms$ (since 32 processors are used)

$\rightarrow 14.2 ms + 1.8 ms = 16 ms$ are needed to solve the problem

$\frac{72ms}{16ms} = 4$

$\rightarrow$ Speedup is 4

\item 
\begin{align*}
	32 &= 72 * x + 0.25 * 72 * (1-x) \\
	32 &= 72 * (x + 0.25 * (1-x)) \\
	32 &= 72 * (x + 0.25 - 0.25x) \\
	32 &= 72 * (0.75x + 0.25) \\
	32 &= 48x + 16 \\
	16 &= 48x \\
	x &= \frac{1}{3} \\
\end{align*}

1/3 der Gesamtzeit (24ms) wird zum initialisieren benoetigt

\item 
maximum speedup with infinite amouts of processors ($S_{max}$) \\
(infinite perfectly parallel working processors reduce the parallel runntime to effectively zero)

\begin{align*}
	S_{max} &= \frac{\text{normal runtime}}{\text{(initialization time + paralellized time)}} \\
	&= \frac{72}{(24 + lim_{x\rightarrow 0} x)} = \frac{72}{(24 + 0)} = 3
\end{align*}

if an infinite number of processors run perfecly parallel then only the initialization time
remains. Therefore the minimal runtime of the program is 24 ms

\item 
\begin{align*}
	x &= 72 * \frac{1}{6} + \frac{1}{32} * 72 * \frac{5}{6} \\
	&= 72 * (\frac{1}{6} + \frac{5}{192}) \\
	&= 72 * (\frac{32}{192} + \frac{5}{192}) \\
	&= 72 * (\frac{37}{192}) \\
	&= \frac{2664}{192} \\
	&=13.875 \\
\end{align*}
$\rightarrow$ With the new algorithm the program fnishes in less then 14 ms and the company is saved!

\end{enumerate}

\end{document}
